%  LatexEx1.tex - An introduction to LaTeX

\documentclass[a4paper, 10pt]{article}

\usepackage{amsmath} % Used for mathematical equations
\usepackage{amssymb} % As above
\usepackage{natbib} % Used for bibliography
\usepackage{graphicx} % Used to include figures
\usepackage{epstopdf} % Convert .eps files to .pdf
\usepackage{geometry}
\usepackage{natbib}
\bibpunct[, ]{(}{)}{;}{a}{,}{,}

\title{An Introduction to \LaTeX}
\author{T.M. McDonald}
\date{24th September 2019}

\begin{document}
\maketitle
\section{Introduction \label{S:Intro}}
This document is an introduction to the use of \LaTeX. We will show how to use \LaTeX\ to
produce most things which would be needed in a short report. We will show how to produce
tables, include figures and typeset some simple mathematics.

\section{Typesetting \label{S:Typesetting}}

\subsection{Changing Text Formats \label{S:Text Formats}}
I had a bowl of \textbf{cereal} for breakfast and a \emph{sandwich} for lunch.

\subsection{Creating Lists \label{S:Lists}}

\subsubsection{Itemize \label{S:Itemize}}
\begin{itemize}
\item Baseball
\item Basketball
\item Cricket
\end{itemize}

\subsubsection{Enumerate \label{S:Enumerate}}
For dinner last night, I had:
\begin{enumerate}
\item Pasta with Arrabiata Sauce
\item Garlic Bread
\end{enumerate}

\section{Mathematics \label{S:Maths}}
\begin{equation}
a = (b+c)^d \label{E:NewEq}
\end{equation}

Add $a$ and $b$ to get $c$, or written more formally $a + b = c$.

Add $a$ and $b$ to get $c$, or written more formally
\begin{equation}
a + b = c. \label{E:SimpleEx}
\end{equation}


\[
\frac{\frac{1}{x}+\frac{1}{y}}{z}
\] 

\[
x^a x^b= x^{a+b}
\] 

\begin{equation*}
\int_{x^2 < 1}^{}
f(x) dx
\end{equation*}

\[
\forall \in \mathbf{R}:	x^2 \geq 0
\]

\begin{align*}
f(x) & = (x-a)(x+a) \\
& = x^2 - ax + ax - a^2 \\
& = x^2 - a^2 \\
\end{align*}

\[
\left(
\begin{array}
{c c}
3 & 1 \\
9 & 2 \\
\end{array}
\right)
\]

\section{Figures \label{S:Figures}}
File was unavailable, more information regarding figures and editing captions etc. from R plot .pdfs is available in the LaTeX notes.

\section{Tables \label{S:Tables}}
\begin{table}
\centering
\begin{tabular}{|l||c|r|} \hline
Name & Exam 1 & Exam 2 \\ \hline
Bob Smith & 43 & 81 \\
Anne Frank & 75 & 70 \\ \hline
\end{tabular}
7
\caption{Exam Marks}
\label{T:Exam Marks}
\end{table}

Exam marks 2 file was unavailable, come back to this later!

\section{Labelling \label{S:Labelling}}
We previously mentioned equation \ref{E:SimpleEx} in section \ref{S:Maths}

\section{Citations \label{S:Citations}}
\bibliographystyle{rmd}
\bibliography{refdatabase}

\section{Complex Tables \label{S:Complex Tables}}
\begin{center}
\begin{tabular}{l@{\hskip 0.5in}|crr@{.}l|}
\multicolumn{1}{l}{Fell race} &  \multicolumn{3}{c}{Data}\\
\cline{2-5}
& dist & climb & \multicolumn{2}{c|}{time} \\
\cline{2-5}
Greenmantle & 2.5 & 650 & 16 & 083 \\
Carnethy & 6.0 & 2500 & 48 & 350 \\
Craig Dunain & 6.0 & 900 & 3 & 650 \\ \cline{2-5}
\end{tabular}
\end{center}

\section{Placing Tables and Figures - Floats \label{S:Floats}}


\end{document}
